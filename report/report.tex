
\documentclass[11pt]{article}
%%%%%%%%%%%%%%%%%%%%%%%%%%%%%%%%
\usepackage{amsmath}
\usepackage{amsthm}
\usepackage{footnote}
\usepackage{fancyhdr}
\usepackage{amssymb}
\usepackage{tikz}

\theoremstyle{plain}
\newtheorem*{theorem*}{Theorem}
\newtheorem*{notation*}{Notation}
\newtheorem*{algorithm*}{Algorithm}
\newtheorem*{remark*}{Remark}
\newtheorem{theorem}{Theorem}[section]
\newtheorem{algorithm}{Algorithm}[section]
\newtheorem{axiom}[theorem]{Axiom}
\newtheorem{claim}[theorem]{Claim}
\newtheorem{corollary}[theorem]{Corollary}
\newtheorem{definition}[theorem]{Definition}
\newtheorem{example}[theorem]{Example}
\newtheorem{lemma}[theorem]{Lemma}
\newtheorem{notation}[theorem]{Notation}
\newtheorem{problem}[theorem]{Problem}
\newtheorem{proposition}[theorem]{Proposition}
\newtheorem{remark}[theorem]{Remark}
\newtheorem{summary}[theorem]{Summary}
\newtheorem{assumption}{Assumption}[section]

\DeclareMathOperator*{\argmax}{arg\,max}
\DeclareMathOperator*{\E}{\mathbb{E}}
\let\Pr\relax
\DeclareMathOperator*{\Pr}{\mathbb{P}}

\begin{document}

\title{Report for Independent Study\\
Deep Learning and Part-of-speech Tagging}
\author{Min Chen}
\date{\today}
\maketitle
\thispagestyle{fancy}
\lhead{Deep Learning and POS}
\rhead{IUB Summer 2018}
\rfoot{Page \thepage}

\begin{abstract}
	
This is the report for an independent study with Prof. Damir Cavar at Indiana 
University Bloomington for the Summer Semester 2018. The independent 
study is conducted in several parts throughout the summer including reading 
groups led by Prof. Damir, guided reading over reference books, slides and 
papers, as well as hands-on coding for part-of-speech tagging using deep 
learning techniques. 

\end{abstract}
\pagebreak
\tableofcontents
\pagebreak


\section{Introduction}

Deep learning has been a hot topic in computer science especially artificial 
intelligence (AI) in recent years, especially after the five-game Go match 
between 18-time world champion Lee Sedol and AlphaGo, a computer Go 
program developed by Google DeepMind. Deep learning or more specifically, 
deep neural networks has been applied to most if not all AI applications such 
as computer vision and natural language processing. 

Prof. Damir Cavar at Indiana University taught a course named Deep Learning 
and NLP in Spring 2017 and I did not have chance to participate. Since the 
techniques and applications of deep learning is so interesting and important, 
I decided to participate in an independent study with Prof. Cavar, focusing on 
deep learning with applications on natural language processing. 


\paragraph{Outline}The report is structured in the following way: 
Part \ref{p:pre} talks about basic streaming models and definitions followed 
by commonly used probability tools in the analysis in this field including 
several inequalities for bounding the tail distributions. It also provides a 
summary of the widely used universal hash family and pseudo-random 
generator for sublinear space. The author tried to unify materials from 
several places and present in a easily understandable way with clear 
indications on the space usage and construction method. Part \ref{p:fm} 
focuses on frequency moments estimation and Part \ref{p:pq-hh} focuses on 
two related problems point query and heavy hitters. In these parts, the 
author unifies the algorithm from several lecture notes and organize them 
logically. Part \ref{p:sr-l0} introduces the important concept of 
$l_0$-sampling which is used as a tool for the graph sketches in Part 
\ref{p:graph}.


\paragraph{Materials and Resources }
There are several important sources that the author read and referenced 
when studying this topic and writing the report. 
\begin{itemize}
	\item Prof. Qin Zhang's course materials and 
	slides  \cite{zhang2017-slides}.The author followed these sets of slides as 
	a guidance for the major topics with a focus two parts, Sublinear in space 
	and Sublinear in communication. 
	\item Lecture Notes by Prof. Amit Chakrabarti  \cite{Cha2015-notes}. The 
	lecture notes provides detailed explanation, correctness proof and 
	discussion of several of the important streaming algorithms. 
	\item Course materials  by Prof. Jelani Nelson  \cite{Nel2015-web}. Prof. 
	Nelson from Harvard provides detailed lecture notes with Youtube videos 
	for recorded lectures. 
	\item Two papers by Ann, Guha and 
	McGregor  \cite{AGM2012-analyzing}  \cite{AGM2012-graph}. These 
	papers discuss a series of algorithms for graph sketch based on a sketch 
	algorithm of connectivity. 
	\item Text book: Probability and Computing by Mitzenmacher and 
	Upfal  \cite{MU-probability}. The book provides detailed explanation with 
	examples illustrating randomized algorithms used in computing. Some of 
	the preliminary probability tools are learnt and summarized from this book.
	\item Other notes, materials and papers used, will be acknowledged and 
	cited in later sections.
\end{itemize}


\pagebreak
\appendix

\bibliographystyle{alpha}
\bibliography{report} 

\end{document}
